\documentclass[]{article}
\usepackage{lmodern}
\usepackage{amssymb,amsmath}
\usepackage{ifxetex,ifluatex}
\usepackage{fixltx2e} % provides \textsubscript
\ifnum 0\ifxetex 1\fi\ifluatex 1\fi=0 % if pdftex
  \usepackage[T1]{fontenc}
  \usepackage[utf8]{inputenc}
\else % if luatex or xelatex
  \ifxetex
    \usepackage{mathspec}
  \else
    \usepackage{fontspec}
  \fi
  \defaultfontfeatures{Ligatures=TeX,Scale=MatchLowercase}
\fi
% use upquote if available, for straight quotes in verbatim environments
\IfFileExists{upquote.sty}{\usepackage{upquote}}{}
% use microtype if available
\IfFileExists{microtype.sty}{%
\usepackage{microtype}
\UseMicrotypeSet[protrusion]{basicmath} % disable protrusion for tt fonts
}{}
\usepackage[margin=1in]{geometry}
\usepackage{hyperref}
\hypersetup{unicode=true,
            pdftitle={CBST Portfolio Examples},
            pdfauthor={Kiri Daust},
            pdfborder={0 0 0},
            breaklinks=true}
\urlstyle{same}  % don't use monospace font for urls
\usepackage{color}
\usepackage{fancyvrb}
\newcommand{\VerbBar}{|}
\newcommand{\VERB}{\Verb[commandchars=\\\{\}]}
\DefineVerbatimEnvironment{Highlighting}{Verbatim}{commandchars=\\\{\}}
% Add ',fontsize=\small' for more characters per line
\usepackage{framed}
\definecolor{shadecolor}{RGB}{248,248,248}
\newenvironment{Shaded}{\begin{snugshade}}{\end{snugshade}}
\newcommand{\KeywordTok}[1]{\textcolor[rgb]{0.13,0.29,0.53}{\textbf{#1}}}
\newcommand{\DataTypeTok}[1]{\textcolor[rgb]{0.13,0.29,0.53}{#1}}
\newcommand{\DecValTok}[1]{\textcolor[rgb]{0.00,0.00,0.81}{#1}}
\newcommand{\BaseNTok}[1]{\textcolor[rgb]{0.00,0.00,0.81}{#1}}
\newcommand{\FloatTok}[1]{\textcolor[rgb]{0.00,0.00,0.81}{#1}}
\newcommand{\ConstantTok}[1]{\textcolor[rgb]{0.00,0.00,0.00}{#1}}
\newcommand{\CharTok}[1]{\textcolor[rgb]{0.31,0.60,0.02}{#1}}
\newcommand{\SpecialCharTok}[1]{\textcolor[rgb]{0.00,0.00,0.00}{#1}}
\newcommand{\StringTok}[1]{\textcolor[rgb]{0.31,0.60,0.02}{#1}}
\newcommand{\VerbatimStringTok}[1]{\textcolor[rgb]{0.31,0.60,0.02}{#1}}
\newcommand{\SpecialStringTok}[1]{\textcolor[rgb]{0.31,0.60,0.02}{#1}}
\newcommand{\ImportTok}[1]{#1}
\newcommand{\CommentTok}[1]{\textcolor[rgb]{0.56,0.35,0.01}{\textit{#1}}}
\newcommand{\DocumentationTok}[1]{\textcolor[rgb]{0.56,0.35,0.01}{\textbf{\textit{#1}}}}
\newcommand{\AnnotationTok}[1]{\textcolor[rgb]{0.56,0.35,0.01}{\textbf{\textit{#1}}}}
\newcommand{\CommentVarTok}[1]{\textcolor[rgb]{0.56,0.35,0.01}{\textbf{\textit{#1}}}}
\newcommand{\OtherTok}[1]{\textcolor[rgb]{0.56,0.35,0.01}{#1}}
\newcommand{\FunctionTok}[1]{\textcolor[rgb]{0.00,0.00,0.00}{#1}}
\newcommand{\VariableTok}[1]{\textcolor[rgb]{0.00,0.00,0.00}{#1}}
\newcommand{\ControlFlowTok}[1]{\textcolor[rgb]{0.13,0.29,0.53}{\textbf{#1}}}
\newcommand{\OperatorTok}[1]{\textcolor[rgb]{0.81,0.36,0.00}{\textbf{#1}}}
\newcommand{\BuiltInTok}[1]{#1}
\newcommand{\ExtensionTok}[1]{#1}
\newcommand{\PreprocessorTok}[1]{\textcolor[rgb]{0.56,0.35,0.01}{\textit{#1}}}
\newcommand{\AttributeTok}[1]{\textcolor[rgb]{0.77,0.63,0.00}{#1}}
\newcommand{\RegionMarkerTok}[1]{#1}
\newcommand{\InformationTok}[1]{\textcolor[rgb]{0.56,0.35,0.01}{\textbf{\textit{#1}}}}
\newcommand{\WarningTok}[1]{\textcolor[rgb]{0.56,0.35,0.01}{\textbf{\textit{#1}}}}
\newcommand{\AlertTok}[1]{\textcolor[rgb]{0.94,0.16,0.16}{#1}}
\newcommand{\ErrorTok}[1]{\textcolor[rgb]{0.64,0.00,0.00}{\textbf{#1}}}
\newcommand{\NormalTok}[1]{#1}
\usepackage{graphicx,grffile}
\makeatletter
\def\maxwidth{\ifdim\Gin@nat@width>\linewidth\linewidth\else\Gin@nat@width\fi}
\def\maxheight{\ifdim\Gin@nat@height>\textheight\textheight\else\Gin@nat@height\fi}
\makeatother
% Scale images if necessary, so that they will not overflow the page
% margins by default, and it is still possible to overwrite the defaults
% using explicit options in \includegraphics[width, height, ...]{}
\setkeys{Gin}{width=\maxwidth,height=\maxheight,keepaspectratio}
\IfFileExists{parskip.sty}{%
\usepackage{parskip}
}{% else
\setlength{\parindent}{0pt}
\setlength{\parskip}{6pt plus 2pt minus 1pt}
}
\setlength{\emergencystretch}{3em}  % prevent overfull lines
\providecommand{\tightlist}{%
  \setlength{\itemsep}{0pt}\setlength{\parskip}{0pt}}
\setcounter{secnumdepth}{0}
% Redefines (sub)paragraphs to behave more like sections
\ifx\paragraph\undefined\else
\let\oldparagraph\paragraph
\renewcommand{\paragraph}[1]{\oldparagraph{#1}\mbox{}}
\fi
\ifx\subparagraph\undefined\else
\let\oldsubparagraph\subparagraph
\renewcommand{\subparagraph}[1]{\oldsubparagraph{#1}\mbox{}}
\fi

%%% Use protect on footnotes to avoid problems with footnotes in titles
\let\rmarkdownfootnote\footnote%
\def\footnote{\protect\rmarkdownfootnote}

%%% Change title format to be more compact
\usepackage{titling}

% Create subtitle command for use in maketitle
\newcommand{\subtitle}[1]{
  \posttitle{
    \begin{center}\large#1\end{center}
    }
}

\setlength{\droptitle}{-2em}

  \title{CBST Portfolio Examples}
    \pretitle{\vspace{\droptitle}\centering\huge}
  \posttitle{\par}
    \author{Kiri Daust}
    \preauthor{\centering\large\emph}
  \postauthor{\par}
      \predate{\centering\large\emph}
  \postdate{\par}
    \date{May 3, 2019}


\begin{document}
\maketitle

\subsection{Brief Description}\label{brief-description}

This document contains a brief description and some examples of using
Markowitz Portfolio Optimization to select the best seed to plant given
predicted climate change. The general process is outlined below:

\begin{enumerate}
\def\labelenumi{\arabic{enumi}.}
\tightlist
\item
  Import genetic suitability information and BGC prediction from CCISS
  tool
\item
  For each potential seed location within each of the 30 models within
  each site, do:

  \begin{itemize}
  \tightlist
  \item
    Create future data by matching each predicted BGC to the
    corresponding set of genetic suitability
  \item
    Calculate possible max growth each year as
    \(growth = (HTpred - 0.9)*10\)
  \item
    Skip seed location if any growths rates \textless{} 0
  \item
    Calculate probability of death as \(1 - growth\) and rescale to
    {[}0.01,0.1{]}
  \item
    Simulate growth over 100 years with 100 trees using the gamma
    distribution for probability of death
  \end{itemize}
\item
  Skip if fewer than 3 possible seed locations for that model
\item
  Run portfolio optimization within each model using all possible seed
  locations, each time with 25 different risk aversion levels
\item
  Remove locations that are only possible in \textless{} 25\% of climate
  models
\item
  Standardize and scale weights and returns
\end{enumerate}

\subsection{Example Growth Prediction}\label{example-growth-prediction}

The below plot shows the interpolated genetic compatibility for all
possilbe seed locations in one site and one model. Each colour
represents a seed from a specific BGC

\subsection[]{\texorpdfstring{\protect\includegraphics{PortExample_files/figure-latex/simulation-1.pdf}}{}}\label{section}

The next plot shows the cumulative growth in each BGC, with added
probability of death

\subsection[]{\texorpdfstring{\protect\includegraphics{PortExample_files/figure-latex/plots-1.pdf}}{}}\label{section-1}

Finaly, we take the derivative of the cumulative growth to obtain the
growth rate (note that currently we're adding little prob of death, so
it looks very similar to the original compatibility)

\subsection[]{\texorpdfstring{\protect\includegraphics{PortExample_files/figure-latex/plots2-1.pdf}}{}}\label{section-2}

The code below shows an example of the portfolio optimisation for a
given allowable risk with the current settings and output

\begin{Shaded}
\begin{Highlighting}[]
\NormalTok{  returns <-}\StringTok{ }\NormalTok{modData}
    \KeywordTok{rownames}\NormalTok{(returns) <-}\StringTok{ }\KeywordTok{paste}\NormalTok{(returns}\OperatorTok{$}\NormalTok{Year,}\StringTok{"-01-01"}\NormalTok{, }\DataTypeTok{sep =} \StringTok{""}\NormalTok{)}
\NormalTok{    returns <-}\StringTok{ }\NormalTok{returns[,}\OperatorTok{-}\DecValTok{1}\NormalTok{]}
\NormalTok{    returnsTS <-}\StringTok{ }\KeywordTok{as.xts}\NormalTok{(returns)}
    
\NormalTok{    init.portfolio <-}\StringTok{ }\KeywordTok{portfolio.spec}\NormalTok{(}\DataTypeTok{assets =} \KeywordTok{colnames}\NormalTok{(returnsTS))}
\NormalTok{    nSpp <-}\StringTok{ }\KeywordTok{length}\NormalTok{(}\KeywordTok{colnames}\NormalTok{(returnsTS))}
\NormalTok{    init.portfolio <-}\StringTok{ }\KeywordTok{add.constraint}\NormalTok{(}\DataTypeTok{portfolio =}\NormalTok{ init.portfolio, }\DataTypeTok{type =} \StringTok{"weight_sum"}\NormalTok{, }\DataTypeTok{min_sum =} \FloatTok{0.9}\NormalTok{, }\DataTypeTok{max_sum =} \FloatTok{1.1}\NormalTok{) ###weights should add to about 1}
\NormalTok{    init.portfolio <-}\StringTok{ }\KeywordTok{add.constraint}\NormalTok{(}\DataTypeTok{portfolio =}\NormalTok{ init.portfolio, }\DataTypeTok{type =} \StringTok{"box"}\NormalTok{, }\DataTypeTok{min =} \KeywordTok{rep}\NormalTok{(}\DecValTok{0}\NormalTok{, nSpp), }\DataTypeTok{max =} \KeywordTok{rep}\NormalTok{(}\FloatTok{0.95}\NormalTok{, nSpp)) ##set min and max weight for each species}
    
\NormalTok{      qu <-}\StringTok{ }\KeywordTok{add.objective}\NormalTok{(}\DataTypeTok{portfolio=}\NormalTok{init.portfolio, }\DataTypeTok{type=}\StringTok{"return"}\NormalTok{, }\DataTypeTok{name=}\StringTok{"mean"}\NormalTok{)##add return objective}
\NormalTok{      qu <-}\StringTok{ }\KeywordTok{add.objective}\NormalTok{(}\DataTypeTok{portfolio=}\NormalTok{qu, }\DataTypeTok{type=}\StringTok{"risk"}\NormalTok{, }\DataTypeTok{name=}\StringTok{"var"}\NormalTok{, }\DataTypeTok{risk_aversion =} \DecValTok{12}\NormalTok{)##minimize risk using varience}
      
\NormalTok{      minSD.opt <-}\StringTok{ }\KeywordTok{optimize.portfolio}\NormalTok{(}\DataTypeTok{R =}\NormalTok{ returnsTS, }\DataTypeTok{portfolio =}\NormalTok{ qu, }\DataTypeTok{optimize_method =} \StringTok{"ROI"}\NormalTok{, }\DataTypeTok{trace =} \OtherTok{TRUE}\NormalTok{)}
\NormalTok{      minSD.opt}
\end{Highlighting}
\end{Shaded}

\begin{verbatim}
## ***********************************
## PortfolioAnalytics Optimization
## ***********************************
## 
## Call:
## optimize.portfolio(R = returnsTS, portfolio = qu, optimize_method = "ROI", 
##     trace = TRUE)
## 
## Optimal Weights:
## ESSFwk1  ICHdw3  ICHmc1  ICHmc2  ICHmk3   ICHmm   ICHvc  ICHwk2  ICHwk3 
##  0.8126  0.0269  0.0000  0.0000  0.0000  0.0605  0.0000  0.0000  0.0000 
##  ICHwk4   SBSvk  SBSwk1 
##  0.0000  0.0000  0.0000 
## 
## Objective Measure:
##  mean 
## 11.74 
## 
## 
## StdDev 
##  2.394
\end{verbatim}

\subsection{Example Output}\label{example-output}

\textbf{The examples here are using points within the SBSmc2 subzone in
the Bulkley TSA for Lodgepole Pine}\\
Efficient frontiers show the optimal weighting of each asset as risk
decreases (i.e.~the weighting that will give you the most return for an
allowable risk). The graphs below show the efficient frontier for six
randomly selected individual locations. Each colour represents a seed
location, and the black line shows the decrease in max return with
decreased risk.

\subsection{\texorpdfstring{\includegraphics[width=1.2\linewidth]{PortExample_files/figure-latex/multiple-1}}{}}\label{section-3}

The below figure is an average efficient frontier over a selection of
points in the Bulkley TSA. To make it more readable, only seed locations
with a max values \textgreater{} 0.05 are included.

\includegraphics{PortExample_files/figure-latex/average-1.pdf}


\end{document}
